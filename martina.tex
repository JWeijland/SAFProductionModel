\documentclass[12pt,a4paper]{article}
\usepackage[utf8]{inputenc}
\usepackage[margin=1in]{geometry}
\usepackage{amsmath}
\usepackage{titlesec}
\usepackage{xcolor}
\usepackage{parskip}

\definecolor{primaryblue}{RGB}{30, 64, 175}

\titleformat{\section}{\Large\bfseries\color{primaryblue}}{\thesection.}{0.5em}{}
\titleformat{\subsection}{\large\bfseries}{\thesubsection}{0.5em}{}

\setlength{\parindent}{0pt}
\setlength{\parskip}{0.8em}

\begin{document}

\title{\textbf{Dynamic Feedstock Pricing Strategy:\\Balancing Investor Participation and Aggregator Profit}}
\author{}
\date{October 23, 2025}
\maketitle

\section{Problem Statement}

A feedstock aggregator operates in a competitive regional market where multiple investors express interest in constructing production plants. The aggregator observes that heightened investment interest typically signals investor expectations of low feedstock prices, creating an opportunity to increase prices and capture additional profit margin.

However, this pricing decision involves a fundamental trade-off. If the aggregator raises the feedstock price excessively, one or more investors may experience reduced Net Present Value (NPV) below their investment threshold, leading them to withdraw from the project. This withdrawal reduces overall feedstock demand and potentially decreases the aggregator's total revenue.

The central challenge is therefore to determine the optimal feedstock price that maximizes the aggregator's profit while maintaining sufficient investor participation. 

\section{Problem Analysis}

The pricing challenge can take some different forms like state below: 

\textbf{Investor Sensitivity Analysis.} Each investor possesses a unique NPV function that depends on feedstock price, production capacity, operational costs, and market conditions. The critical parameter is the maximum acceptable feedstock price beyond which the investor's NPV becomes negative or falls below their investment hurdle rate. Identifying these threshold prices for all potential investors is essential for informed pricing decisions.

\textbf{Market Equilibrium Identification.} The aggregator must identify a price point where total profit is maximized subject to the constraint that all desired investors remain active. 


\textbf{Strategic Interaction} In low feedstock circumstances, the aggregator and the investor can make agreements in favour for eachother in the form of contracts. The Aggragator can ask a higher price for the investor that wants to make sure he will get the feedstock before other plants does. 

\section{Proposed Solutions}

Four distinct methodological approaches are proposed to address the feedstock pricing problem. 

\subsection{Solution 1: Iterative Price Adjustment Model}

\textbf{Conceptual Approach.} The aggregator implements a feedback-based pricing mechanism that gradually adjusts the feedstock price in response to observed investor interest. When the number of active investors exceeds a target threshold, the price increases step by step. When investor participation falls below the threshold, the price decreases. This process continues until a stable equilibrium is reached. This method reflects a simple and realistic approach. 

\textbf{Implementation in Code.} The implementation requires maintaining a database of potential investors with their calculated NPV functions and price thresholds. The algorithm operates as follows:

Initialize the feedstock price at a conservative baseline level. For each iteration, evaluate the NPV of all potential investors at the current price. Count the number of investors with positive NPV above their NPV threshold. If the active investor count exceeds the target level, increase the price by a predetermined step size. If the count falls below the target, decrease the price. This will create a dynamic that will attract or push away investors based on the target number of investor. 



\subsection{Solution 2: Auction-Based Mechanism}

\textbf{Conceptual Approach.} This approach treats feedstock pricing as an  auction where investors reveal their willingness to pay through their investment decisions. The aggregator requests investors to indicate their maximum acceptable feedstock prices, either explicitly through bids or implicitly through revealed preference. The aggregator then sets the price just below the lowest acceptable threshold among selected investors, maximizing revenue while ensuring all targeted investors remain viable. This seems a bit unfair because this would squeeze the investors maximally assuming total knowledge. 

\textbf{Implementation in Code.} The implementation requires a bidding or preference revelation mechanism integrated into the investor evaluation system.

First, collect or estimate each investor's maximum acceptable feedstock price based on their NPV calculations. Store these threshold prices in a sorted array. Determine the desired number of investors to retain based on market capacity and strategic objectives. Set the feedstock price equal to the threshold of the marginal investor minus a small buffer to ensure participation.

The code structure includes a threshold collection module that either receives explicit bids or computes thresholds from investor financial models. A sorting and selection algorithm identifies the optimal price point. The profit calculation module then determines total revenue and margin based on the selected price and resulting demand.

\subsection{Solution 3: Optimization-Based Mathematical Model}

\textbf{Conceptual Approach.} This approach formulates the pricing problem as a mathematical optimization model where the aggregator's profit is the objective function subject to constraints representing investor participation conditions. The model explicitly captures the relationship between feedstock price, investor NPV, and participation decisions. By solving this optimization problem, the aggregator identifies the theoretically optimal price. This is a bit of an overkill, although I really am pro MILP.

\textbf{Implementation in Code.} The implementation requires formulating and solving a constrained optimization problem using standard optimization libraries.

Define the objective function as aggregator profit: the feedstock price multiplied by the sum of demand from all participating investors. Define binary decision variables indicating whether each investor participates. Add constraints ensuring that an investor participates only if their NPV at the chosen price exceeds their threshold and more. Include upper and lower bounds on the feedstock price based on market conditions.

The code structure initializes an optimization model object using a solver such as Gurobi. Decision variables are created for the feedstock price and investor participation indicators. Constraints linking investor NPV to participation decisions are added as logical conditions. The solver is then invoked to find the optimal solution. Results are extracted to determine the recommended feedstock price and expected active investors.


\section{Mathematical Formulation for Solution 1}

To implement the iterative price adjustment model, the following mathematical formulas provide a structured framework for the algorithm.

\textbf{Investor NPV Function.} For each investor $i \in \{1, 2, \ldots, n\}$, the Net Present Value is calculated as:
\begin{equation}
\text{NPV}_i(p) = a_i - b_i \cdot p
\end{equation}
where $p$ represents the feedstock price, $a_i$ is the investor's revenue potential (baseline NPV at zero feedstock cost), and $b_i$ is the price sensitivity coefficient reflecting how NPV decreases with increasing feedstock price.

\textbf{Investor Participation Condition.} An investor participates if their NPV exceeds their minimum investment threshold $\theta_i$:
\begin{equation}
\text{Participate}_i =
\begin{cases}
1 & \text{if } \text{NPV}_i(p) \geq \theta_i \\
0 & \text{otherwise}
\end{cases}
\end{equation}

% \textbf{Total Demand Function.} The total feedstock demand at price $p$ is the sum of individual demands from participating investors:
% \begin{equation}
% D(p) = \sum_{i=1}^{n} d_i \cdot \text{Participate}_i(p)
% \end{equation}
% where $d_i$ represents the feedstock demand (tons per year) from investor $i$.

% \textbf{Aggregator Profit Function.} The aggregator's total profit is:
% \begin{equation}
% \Pi(p) = p \cdot D(p)
% \end{equation}

\textbf{Proportional Step Size.} Let the step size scale with the mismatch:
\begin{equation}
\Delta p_t = - \alpha \cdot (N_{\text{active}}(p_t) - N_{\text{target}} )
\end{equation}


Where N is the number of investors. This formula is comparable to the formula used for the update of the DR (Discount Rate). The alpha should be in a well defined order of magnitude. 





\textbf{Iterative Price Update Rule.} The feedstock price is adjusted at each iteration $t$ based on the number of active investors:
\begin{equation}
p_{t+1} = p_t + \Delta p_t
\end{equation}
where $\Delta p$ is the price adjustment step size.




\section{Mathematical Formulation for Solution 2}

\textbf{Investor Threshold Price.} For each investor $i \in \{1, 2, \ldots, n\}$, define the maximum acceptable feedstock price:
\begin{equation}
p_i^{\max} = \frac{a_i - \theta_i}{b_i}
\end{equation}
where $a_i$ and $b_i$ are as defined before, and $\theta_i$ is the investor's NPV threshold.

\textbf{Sorted Thresholds.} Sort all $p_i^{\max}$ in descending order:
\begin{equation}
p_{(1)}^{\max} \geq p_{(2)}^{\max} \geq \dots \geq p_{(n)}^{\max}
\end{equation}

\textbf{Price Selection.} Let $k$ be the target number of investors to retain. The feedstock price is set as:
\begin{equation}
p^* = p_{(k)}^{\max} - \varepsilon
\end{equation}
where $\varepsilon > 0$ is a small buffer to ensure participation.



\section{Mathematical Formulation for Solution 3}

\textbf{Decision Variables:}
\[
p \geq 0 \quad \text{(feedstock price)}, \qquad x_i \in \{0,1\} \; (i = 1,\dots,n)
\]

\textbf{Objective:}
\[
\max \; \Pi = p \cdot \sum_{i=1}^{n} d_i x_i
\]

\textbf{Investor Maximum Price:}
\[
p_i^{\max} = \frac{a_i - \theta_i}{b_i}, \qquad \text{from } a_i - b_i p \geq \theta_i
\]

\textbf{Big-M Constraint:}
\[
a_i - b_i p \geq \theta_i - M(1 - x_i), \quad \forall i
\]

If $x_i = 1$, the condition enforces $a_i - b_i p \geq \theta_i$.  
If $x_i = 0$, the right-hand side becomes very negative, so the constraint is inactive.


\section*{Disclaimer}
This document was prepared with the assistance of AI tools (e.g., ChatGPT) for drafting and structuring content. All content has been thoroughly reviewed, verified, and approved by the author, who assumes full responsibility for its accuracy and integrity.

\end{document}

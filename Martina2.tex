\documentclass[12pt,a4paper]{article}
\usepackage[utf8]{inputenc}
\usepackage[margin=1in]{geometry}
\usepackage{amsmath}
\usepackage{titlesec}
\usepackage{xcolor}
\usepackage{parskip}

\definecolor{primaryblue}{RGB}{30, 64, 175}

\titleformat{\section}{\Large\bfseries\color{primaryblue}}{\thesection.}{0.5em}{}
\titleformat{\subsection}{\large\bfseries}{\thesubsection}{0.5em}{}

\setlength{\parindent}{0pt}
\setlength{\parskip}{0.8em}

\begin{document}

\title{\textbf{Exploring Feedstock--Investor Dynamics:\\Contractual Agreements with Iterative Price Discovery}}
\author{Jelle Weijland}
\date{October 23, 2025}
\maketitle

\section{Revised Problem Statement}

After reassessing the relationship between the feedstock aggregator and the investor, the idea of adjusting feedstock prices annually based on market dynamics has been reconsidered. The original assumption was that prices would change each year in response to market forces. However, in practice, price fluctuations are mainly caused by transport costs, logistics, and other operational factors rather than pure supply and demand.

This makes an annual market-based pricing approach unrealistic and too unstable for long-term investments. Both aggregators and investors need predictable prices to make confident, long-term financial and operational decisions.

The revised approach therefore focuses on creating a contractual framework that ensures long-term price stability while still allowing for a fair way to set the initial price. The main goal is to find an equilibrium price to formalize this within a long-term agreement.

\section{Problems with the New Statement}

The transition from annual market-based pricing to long-term contractual agreements introduces several critical challenges that must be addressed systematically.

\textbf{Initial Price Discovery.} Since there is no existing market mechanism, finding the initial contract price requires a clear and fair method that works for both aggregators and investors. The price must be high enough for the aggregator to make a profit, but low enough to keep investors interested.  The step size of the priced adjustment and initial price at which the dynamic is beginning is yet to be determined.

\textbf{Competition Between Investors.} Several investors may compete for feedstock from the same aggregator. The aggregator needs to divide its available feedstock and set a price that maximizes total revenue without excluding too many investors. This requires balancing the price level with the number of participants.

\textbf{Long-Term Risk.} A ten- to thirty-year contract carries significant commitment risks for both sides. Investors lock in feedstock costs that affect their NPV for decades, while aggregators commit to long-term supply at fixed or indexed prices. If the aggregator fails to deliver, it may face substantial penalties.

\textbf{Investor Insolvency Risk.} A particularly important risk the aggregator assumes is investor bankruptcy before contract expiration. If an investor becomes insolvent, they may be unable to fulfill their contractual payment obligations, leaving the aggregator with unused capacity that was reserved for that contract. This represents a fundamental risk inherent to long-term agreements and should be factored into the aggregator's risk management strategy. Potential mitigations include requiring financial guarantees, credit assessments, or diversifying across multiple investors to reduce exposure to individual defaults.

\textbf{Price Adjustment Over Time.} Although the contract aims for long-term price stability, it must also include flexibility to account for inflation and changing economic conditions. Whether the price increases by a fixed percentage, follows inflation, or varies within defined upper and lower limits, this decision determines how risk is shared between both parties.


\textbf{Iterative Price Convergence.} During the thirty-day period in which the price is gradually adjusted, both sides must actively respond. Investors will recalculate their NPVs as prices change, while the aggregator monitors participation. The mechanism must reach a stable price within the timeframe without causing volatility or strategic manipulation.
\section{Proposed Approach}

To address the challenges described earlier, a more practical approach is proposed: an iterative price discovery mechanism followed by long-term contracts based on the final equilibrium price. This method aims to stability for both investors and aggregators.

\subsection{Contractual Framework}

Investors interested in building production plants within the area of a specific feedstock aggregator take part in a joint price-setting process with that aggregator. This process takes place once a year during a thirty-day period, giving both sides time to negotiate and confirm feedstock contracts.

When the final price is reached through the iterative mechanism (explained below), the "winning" investor signs a long-term contract with the aggregator. These contracts include:

\textbf{Contract Duration.} The agreement lasts between ten and thirty years. This gives investors a secure feedstock supply and provides aggregators with predictable demand.

\textbf{Price Structure Options.} The price can be structured in different ways:
\begin{itemize}
    \item \textbf{Fixed with Annual Increase:} The base price increases each year by a fixed percentage (e.g., 2\%) to reflect inflation and cost growth.
    \item \textbf{Inflation-Linked:} The price follows an official inflation index such as the CPI, keeping the real value stable over time.
    \item \textbf{Variable with Limits:} The price can move within upper and lower limits (e.g., $\pm$10\%) based on cost indicators but cannot go beyond these bounds.
\end{itemize}

\textbf{Priority Access.} Contracted investors receive priority feedstock access throughout the agreement, ensuring supply stability for long-term operations.



\subsection{Iterative Price Discovery Mechanism}

The key element of this approach is the iterative price discovery process. It runs during the thirty-day transition period each year and helps the price naturally settle at a fair level that benefits both investors and the aggregator.

\textbf{Basic Logic.} The aggregator starts with a target number of investors based on available feedstock capacity. If too many investors want to participate at the current price, the price increases. Some investors will then drop out as their expected returns fall below their NPV threshold. If too few investors remain, the price decreases to attract more participants.  

This daily feedback loop continues until the number of active investors matches the aggregator’s target. The process finds a balance between a sustainable price and the desired participation level.

\textbf{Daily Steps.}
\begin{enumerate}
    \item The aggregator announces the current price.
    \item Investors calculate their NPV and decide whether to stay in or drop out.
    \item The aggregator counts how many investors remain active.
    \item If there are too many investors, the price goes up by a set amount.
    \item If there are too few, the price goes down.
    \item If the number matches the target, the price stays the same.
\end{enumerate}

This continues daily for thirty days or until the price stabilizes.

\textbf{Finalization.} At the end of the process, the final price is set. All investors still active at that price sign long-term contracts using one of the pricing options described earlier. This ensures that everyone who remains in the deal has a positive NPV and that the aggregator reaches its capacity target at a fair and sustainable price.

\textbf{Capacity Management.} A critical consideration arises when all remaining investors at the final price wish to proceed: if they collectively require the aggregator's full capacity for 30 years, no room remains for future contracts. This would prevent the aggregator from potentially benefiting from better economic conditions in later years or diversifying its client base. For this reason, introducing $N_{\text{max,annual}}$ as a constraint on the maximum number of new investors per year is essential. This parameter allows the aggregator to strategically reserve capacity for future opportunities while still maximizing current revenue.



\textbf{Avoiding Manipulation, securing transparancy.} Because the process is transparent and repeated daily, it’s difficult for either side to manipulate the system. Investors can’t easily bluff about their profitability, and aggregators can’t push prices too high without losing participants. The result is a fair price that reflects actual market conditions.

\section{Mathematical Formulation of Solution 1}

The iterative price discovery can be expressed mathematically to show how prices and participation evolve over time.

\textbf{Investor NPV.}
\begin{equation}
\text{NPV}_i(p) = a_i - b_i \cdot p
\end{equation}
where $a_i$ is the investor’s potential NPV with zero feedstock cost and $b_i$ represents how sensitive that NPV is to price.

\textbf{Participation.}
\begin{equation}
\text{Active}_i(p) =
\begin{cases}
1 & \text{if } \text{NPV}_i(p) \geq \theta_i \\
0 & \text{otherwise}
\end{cases}
\end{equation}

\textbf{Total Active Investors.}
\begin{equation}
N_{\text{active}}(p) = \sum_{i=1}^{n} \text{Active}_i(p)
\end{equation}

\textbf{Aggregator Target.}
\begin{equation}
N_{\text{target}} = \min\left(\left\lfloor \frac{Q_{\text{total}}}{\bar{d}} \right\rfloor, N_{\text{max,annual}}\right)
\end{equation}
where $\bar{d}$ is the average feedstock demand per investor and $N_{\text{max,annual}}$ is the maximum number of new contracts allowed per year. This upper limit prevents the aggregator from fully booking all capacity at once, which would leave no room for future contracts under potentially more favorable economic conditions. The actual plant capacity per investor is typically around 100,000 tons annually, making a simple division by total capacity unrealistic without this constraint.

\textbf{Price Update Rule.}
\begin{equation}
p_{k+1} =
\begin{cases}
p_k + \Delta p & \text{if } N_{\text{active}}(p_k) > N_{\text{target}} \\
p_k - \Delta p & \text{if } N_{\text{active}}(p_k) < N_{\text{target}} \\
p_k & \text{otherwise}
\end{cases}
\end{equation}

The process continues until the price change between days becomes very small:
\begin{equation}
|p_{k+1} - p_k| < \epsilon
\end{equation}

Once the equilibrium price $p^*$ is found, it is used in the long-term contract with one of the following formulas:

\textit{Fixed annual increase:}
\begin{equation}
p_t = p^* (1 + g)^{t-1}
\end{equation}

\textit{Inflation-linked:}
\begin{equation}
p_t = p^* \cdot \frac{\text{CPI}_t}{\text{CPI}_0}
\end{equation}

\textit{Variable within bounds:}
\begin{equation}
p_t = \max(p_{\text{floor}}, \min(p_{\text{ceiling}}, p^* f_t))
\end{equation}

\section{Alternative Approaches}

Two other methods were considered as alternatives to the iterative process.

\subsection{Auction-Based Price Discovery}

In this method, investors submit bids showing the maximum price they are willing to pay. The aggregator then selects the top $N_{\text{target}}$ investors and sets the contract price at the lowest accepted bid.  

\textbf{Pros:} Fast and simple to implement.  
\textbf{Cons:} Encourages strategic bidding and may be viewed as unfair if it drives prices too high.

\subsection{MILP Optimization}

This method uses mathematical optimization to select investors and set prices to maximize the aggregator’s profit under capacity and NPV constraints.  

\textbf{Pros:} Produces an optimal result in theory.  
\textbf{Cons:} Requires full knowledge of each investor’s data and complex computational tools, which may not be realistic in practice.

\section{Conclusion and Recommendation}

Among the three approaches, the iterative mechanism is the most balanced and practical option. It is transparent, fair, and easy to implement. It does not require investors to reveal sensitive financial details and naturally finds a stable, fair price based on real participation.  

The auction-based method may lead to unfair outcomes and distrust, while the optimization model is too complex and data-heavy for real-world use.  

Therefore, the recommended solution is the iterative price discovery mechanism combined with long-term contracts that include inflation adjustments or upper and lower price limits. This setup ensures fairness, operational simplicity, and stability for both aggregators and investors.







\section{Contract Expiration and Capacity Re-Entry Dynamics}

When investor contracts reach their end-of-life (typically after 10--30 years), the associated production plants cease operations under the existing agreement. This creates an immediate capacity vacancy within the aggregator's supply network. However, the physical infrastructure remains intact, presenting an opportunity for capacity re-entry under more favorable economic conditions.

\subsection{End-of-Life Mechanism}

\textbf{Contract Termination.} Upon contract expiration, the investor's contractual obligations and priority access rights terminate. The feedstock capacity previously allocated to that investor becomes immediately available for reallocation. If multiple contracts expire simultaneously, the aggregator may face a sudden surplus of available capacity.

\textbf{Re-Entry Opportunity.} The existing plant infrastructure represents a significant advantage for new investors or for the original investor to re-enter under a new contract. Since the capital costs of plant construction have already been absorbed by the previous investor, new entrants face substantially lower barriers to entry. This reduced capital requirement translates to improved NPV calculations, allowing these investors to potentially accept higher feedstock prices than would be feasible for greenfield projects.

\textbf{Competitive Advantage for Existing Plants.} Investors interested in utilizing existing plant capacity enter the same iterative price discovery mechanism described in Section 3.2, but with fundamentally different economics. Their NPV calculations reflect:
\begin{itemize}
    \item Zero or minimal capital expenditure (assuming plant acquisition at depreciated value or lease arrangements)
    \item Reduced operational startup costs
    \item Immediate production capability without construction delays
\end{itemize}

This allows them to remain competitive at higher feedstock prices, effectively increasing the equilibrium price the aggregator can achieve for that capacity.

\textbf{Market Re-Entry Timing.} The aggregator must decide whether to immediately offer expired capacity in the next annual price discovery round or to strategically time re-entry based on market conditions. This decision depends on:
\begin{itemize}
    \item Current demand for feedstock contracts
    \item Economic conditions favoring higher prices
    \item Strategic capacity management for portfolio optimization
\end{itemize}

\subsection{Implementation Requirements}

To incorporate end-of-life dynamics into the existing model, the following code modifications and new mechanisms are required:

\textbf{Contract Tracking System.}
\begin{itemize}
    \item Add a contract database storing: investor ID, contract start year, contract duration, annual capacity allocation, and plant infrastructure status
    \item Implement an annual contract expiration check that identifies contracts reaching end-of-life in the current simulation year
    \item Upon expiration, flag the associated capacity as "available" and mark the plant as "existing infrastructure"
\end{itemize}

\textbf{Differentiated Investor Classes.}
\begin{itemize}
    \item Introduce two investor types: \textit{greenfield investors} (building new plants) and \textit{brownfield investors} (utilizing existing plants)
    \item Modify NPV calculation function to accept an infrastructure status parameter:
    \begin{equation}
    \text{NPV}_i(p, \text{status}) =
    \begin{cases}
    a_i - b_i \cdot p - \text{CAPEX}_{\text{new}} & \text{if greenfield} \\
    a_i - b_i \cdot p - \text{CAPEX}_{\text{existing}} & \text{if brownfield}
    \end{cases}
    \end{equation}
    where $\text{CAPEX}_{\text{existing}} \ll \text{CAPEX}_{\text{new}}$
    \item Brownfield investors have higher $a_i$ values due to reduced capital requirements, allowing participation at higher price points
\end{itemize}

\textbf{Modified Price Discovery Process.}
\begin{itemize}
    \item When available capacity includes both new capacity and expired capacity with existing infrastructure, the aggregator runs separate or combined price discovery rounds
    \item Brownfield investors compete for specific existing plant capacity, while greenfield investors compete for new capacity allocation
    \item Alternatively, both investor types participate simultaneously, with brownfield investors naturally achieving higher NPVs at equivalent prices
\end{itemize}

\textbf{Capacity Reallocation Logic.}
\begin{itemize}
    \item Update the aggregator's target calculation to include both new capacity and newly available expired capacity:
    \begin{equation}
    N_{\text{target}} = \min\left(\left\lfloor \frac{Q_{\text{new}} + Q_{\text{expired}}}{\bar{d}} \right\rfloor, N_{\text{max,annual}}\right)
    \end{equation}
    \item Track capacity by type (new vs. expired) to enable differentiated pricing strategies if desired
\end{itemize}

\textbf{Long-Term Portfolio Dynamics.}
\begin{itemize}
    \item Implement rolling contract management where the aggregator maintains a portfolio of contracts at various stages of their lifecycle
    \item Add visualization capabilities showing contract expiration timelines and capacity availability projections
    \item Enable scenario analysis showing how staggered contract expirations impact long-term revenue and capacity utilization
\end{itemize}

\textbf{Strategic Capacity Management.}
\begin{itemize}
    \item Allow the aggregator to optionally withhold some expired capacity from immediate re-entry, maintaining strategic reserves for future favorable market conditions
    \item Implement a capacity reservation parameter $Q_{\text{reserved}}$ that reduces available capacity in the target calculation
\end{itemize}

This end-of-life mechanism creates a more realistic long-term market dynamic where infrastructure depreciation, capacity cycling, and reduced barriers to re-entry all influence equilibrium prices and investor participation over multi-decade timescales.


\end{document}


